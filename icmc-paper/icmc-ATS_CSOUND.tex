%% LyX 1.3 created this file.  For more info, see http://www.lyx.org/.
%% Do not edit unless you really know what you are doing.
\documentclass[10pt,letterpaper,english]{article}
\usepackage{times}
\usepackage[T1]{fontenc}
\usepackage[latin1]{inputenc}
\usepackage{graphicx}

\makeatletter
%%%%%%%%%%%%%%%%%%%%%%%%%%%%%% User specified LaTeX commands.
\usepackage{fullpage}
\usepackage{chicago}                    % for "(Author, year)" cite style.
\usepackage{indentfirst}                % indent para after headings.
% alex put this here
\usepackage{graphicx}
\setlength{\textheight}{10in}
% \usepackage{url}                      % (handy if you reference a URL.)

\setlength{\oddsidemargin}{-0.25in}     % Latex has one-inch "driver margin"
% \setlength{\evensidemargin}{-0.25in}  % (shouldn't be necessary)
\setlength{\textwidth}{7in}             % 8.5 - 2*0.75 

\setlength{\columnsep}{0.25in}
\setlength{\parindent}{0.2in}

\raggedbottom                           % better than inter-para spaces, I say.



\usepackage{babel}
\makeatother
\begin{document}
\twocolumn


\title{\textbf{ATS user interfaces}}


\author{William 'Pete' Moss and Alex Norman\\
 Center for Digital Arts and Experimental Media, University of Washington\\
 petemoss@u.washington.edu\\
 School of Computer Science and Engineering, University of Washington\\
 alexnorman@users.sourceforge.net\\
 }

\maketitle
\pagestyle{empty} \thispagestyle{empty}


\section{ATS Overview}

The ATS system was first developed by Juan Pampin as a model for representing
sound as a combination of sinusoidal trajectories and bandlimited
noise. The analysis model consists of temporal frames, each of which
contains a set of partials having amplitude and frequency values,
and possibly phase information. Each frame might also contain noise
information, modeled as time-varying energy in the 25 critical bands
of the analysis residual (i.e. the residual is what's left after subtracting
the tracked sinusoidal trajectories from the original sound. Please
visit atsa.sourceforge.net for detailed information on the sinusoids
plus critical-band noise model).


\section{Csound Interface}
The ATS team has written several csound unit generators that extract data from ATS files.  Many of these u-gens are derived from, and work much like Richard Karpen's phase vocoder unit generators for csound (pvread, pvadd etc.).  As with pvread, most of the ats-csound u-gens take a "time pointer" that is used to index data from ATS data files.  Linear interpolation is used to approximate data between analysis frames.

The csound unit generators can be broken down into two groups: those that act independently, and those that depend on the unit generator atsbufread.

\subsection{Independent Unit Generators}
All of the independent u-gens take an ATS file and a time pointer as arguments.  Using these and other arguments, the u-gens extract data from an ATS file at the time indicated by the time pointer.

\textbf{atsread} and \textbf{atsreadnz} are the most generic u-gens in the ats-csound package.  Each of these u-gens simply read data (with interpolation) out of an ATS data file and return it for arbitrary use.  atsread returns the frequency and amplitude information of a user specified partial.  atsreadnz takes a noise band number and returns the corresponding noise energy data.

\textbf{atsadd} and \textbf{atsaddnz} use data from ATS analysis files to synthesize sine-waves or noise respectively.  atsadd uses a standard interpolated table look-up synthesis method to synthesize an array of sinusoids.  These are combined additively to produce a single audio rate output.  The user specifies a range of partials to synthesize.  The frequency and amplitude information for these partials are taken from a given ATS data file.  A frequency multiplier, given by the user, is used before synthesis to transpose the data in frequency.

Like pvadd, atsadd provides an optional amplitude "gate" function.  This "gate" function is given by a csound f-table and is used to scale the partials' amplitudes.  Amplitudes are normalized based on the maxim amplitude in the analysis file.  The gate function uses these normalized amplitudes to index the provided f-table.  A partial with an amplitude of 0 will index the first data point in the f-table.  A partial with the maximum amplitude in the analysis file will index the last position in the f-table.  The value of the indexed data is used to scale the amplitudes before synthesis.  With this "gate" function a user can distort the re-synthesis in subtle or extreme ways.

atsaddnz synthesizes a set of noise bands indicated by the user.  It does this using an array of band limited noise bands that are modulated by sine waves to be put into the correct place in the frequency spectrum.  These bands are combined additively to produce a single audio rate output.  The range of bands in the synthesis is determined by the user.

\textbf{atssinnoi} synthesizes sine waves and noise together in a unique manner.  Partials are synthesized using an internal oscillator.  A noise band is synthesized for each partial.  The energy for each noise band derived from the energy in the analysis noise bands.  Each partial falls into a specific noise band.  The noise energy for each band is distributed equally among the partials in that band.  This is the energy used to synthesize the noise band for each partial.  The bandwidth of a partial's noise band is determined by a function of the partial's frequency.  Once the noise bands have been synthesized they are modulated by their corresponding partial.  Finally, after being scaled appropriately, all of the sine-waves and noise bands are added together to give a single audio rate output.

As with atsadd a frequency multiplier can be used to transpose the data in frequency.  Using provided k-rate inputs, a user can scale the total amplitude of the sine-waves and noise independently after the noise modulation occurs.

The final independent unit generator \textbf{atsbufread} has no outputs that are directly accessible by the user.  atsbufread takes data from an ATS file and produces a table of partials' amplitude and frequency values in memory.  This memory can be accessed by other unit generators.  Like atsadd, atsbufread uses a time pointer and a users specified list of partials to produce the table.  The next section will describe the unit generators that operate on data provided by atsbufread.

\subsection{atsbufread Dependent Unit Generators}
\textbf{atsinterpread} takes a single argument, a frequency value.  Using this frequency value atsinterpread indexes a table produced by an atsbufread u-gen and returns the corresponding amplitude value.  Interpolation is used for in-between values.  This u-gen can be useful for cross synthesizing non ATS derived signals with data from ATS.

\textbf{atspartialtap} works almost exactly like atsread except it reads data from a table produced by an atsbufread.  The only argument this u-gen takes is a partial number; it returns frequency and amplitude data.  This u-gen is useful if a user wants to operate on multiple partials separately using the same time pointer.  While this can easily be achieved with an array of atsreads all using the same time pointer, its simplicity makes it attractive.

\textbf{atscross}, based on pvcross, allows a user to perform cross synthesis using the data from two ATS files.  One of these ATS files comes from the atsbufread, the other is provided by the atscross unit generator.  Data is extracted from the ATS file indicated by the atscross u-gen, and used to index the table produced by the atsbufread u-gen in the same way as atsinterpread indexes the table.  Now each partial from the ATS file provided by the atscross u-gen has two amplitudes; the original amplitude from its data file and the amplitude interpolated from the ATS file of the corresponding atsbufread u-gen.  Each of these amplitudes is then scaled independently by user specified k-rate values.  The amplitudes are then summed and used, with the frequency of the corresponding partial for synthesis.  Synthesis is then achieved using an internal oscillator.  The independent amplitude scalars can be varied to achieve morphing from a standard atsadd sound to a cross-synthesized sound.

\section{Pd Interface}
In order to access ATS data in real time we have created an object for Pure-Data.  This object allows users to access data in a similar way to the csound unit generators atsread and atsreadnz.  This pd object is called \textbf{atsread}.

\textbf{atsread} uses a time pointer the same way that its csound counter part does.  The pd version combines the functionality of the csound atsread and atsread noise.  In addition to it's combined functionality the pd object can output lists of data, allowing for one object to output all the partial and noise data for one file.  atsread for pd has three outputs: the first output gives a list of frequency values, the second gives a list of corresponding amplitude values, and the third gives a list of noise energy values.  The lists are always given in order from lowest partial/band to highest partial/band.  The user opens an ATS file for access by sending a message in this format: "open cl.ats", alternatively the ATS file name can be given as an argument to the object.  A user can set a range of partials by giving messages of the form "set 1 2 5 10..20".  This example would output partials 1, 2, 5, and 10-20 (if they exist in the data file).  Partials can be added to be output using messages of the form "add 3 4..7 9".  This would add partials 3, 4-7 and 9 to be output, and continue outputting the partials that were being output before.  Finally partials can be removed from the output by using messages of the form "remove 4..6 11 13".  The same operations can be performed on noise bands by appending an "nz" to the first argument of the message, IE. "setnz 1 2 17".

The pd atsread object can also be made to only output noise or sine-waves using the "nosines", "nonoise", "sines" and "noise"" messages.  "nosines" turns off frequency and amplitude outputs, "sines" turns them back on.  Similarly "nonoise" turns off noise outputs; "noise" turns them back on.  These messages can also be given as arguments to the object.

As with the csound atsread function the user is left to do what she/he wants with the data.  Synthesis can be preformed with a bank of oscillators, though the data could also be used as control data for other operations. 

\section{ATSH}

ATSH is the graphical editor for ats files. Besides being capable
of displaying ATS information in several different ways, ATSH can
also be used for the analysis, transformation, and synthesis of any
sound.


\subsection{Analysis}

Analysis parameters must be carefully tuned for the analysis algorithm
to properly capture the nature of the signal to be analyzed. As there
are a significant number of them, ATSH offers the possibility of saving
and loading them in a Binary File carrying the extension \char`\"{}{*}.apf\char`\"{}.
The extension is not mandatory, but recommended. A brief explanation
of each Analysis Parameters follows:

\begin{enumerate}
\item Start (secs.): the starting time of the analysis in seconds.
\item Duration (secs.): the duration of the analysis in seconds. A zero
means the whole duration of the input sound file.
\item Lowest Frequency (Hz.): this parameter will partially determine the
size of the Analysis Window to be used. To compute the size of the
Analysis Window, the period of the Lowest Frequency in samples (SR
/ LF) is multiplied by the number of cycles of it the user wants to
fit in the Analysis Window (see parameter 6). This value is rounded
to the next power of two to determine the size of the FFT for the
analysis. The remaining samples are zero-padded. If the signal is
a single, harmonic sound, then the value of the Lowest Frequency should
be its fundamental frequency or a sub-harmonic of it. If it is not
harmonic, then its lowest significant frequency component may be a
good starting value.
\item Highest Frequency (Hz.): highest frequency to be taken into account
for Peak Detection. Once it is determined that no relevant information
is found beyond a certain frequency, the analysis may be faster and
more accurate setting the Highest Frequency parameter to that value.
\item Frequency Deviation (Ratio): frequency deviation allowed for each
peak in the Peak Continuation Algorithm, as a ratio of the frequency
involved. For instance, considering a peak at 440 Hz and a Deviation
of .1 will produce that the Peak Continuation Algorithm will only
try to find candidates for its continuation between 396 and 484 Hz
(10\% above and below the frequency of the peak). A small value is
likely to produce more trajectories whilst a large value will reduce
them, but at the cost of rendering information difficult to be further
processed.
\item Number of Cycles of Lowest Frequency to fit in Analysis Window: this
will also partially determine the size of the Fourier Analysis Window
to be used. See Parameter 3. For single harmonic signals, it is supposed
to be more than one (typically 4).
\item Hop Size (Ratio): size of the gap between one Analysis Window and
the next expressed as a ratio of the Window Size. For instance, a
Hop Size value of .25 will produce an Analysis Window of 2048 samples
to \char`\"{}jump\char`\"{} by 512 samples (Windows will overlap for
a 75\% of their size). This parameter will also determine the size
of the analysis frames obtained. Signals that change their spectra
very fast (such as Speech sounds) may need a high frame rate in order
to properly track their changes.
\item Amplitude Threshold (dB): the highest amplitude value to be taken
into account for Peak Detection.
\item Window Type: the shape of the smoothing function to be used for the
Fourier Analysis. There are four choices available at present: Blackman,
Blackman-Harris, Von Hann, and Hanning. Precise specifications about
them are easily found on D.S.P. bibliography.
\item Track Length (Frames): The Peak Continuation Algorithm will \char`\"{}look-back\char`\"{}
by Length frames in order to do its job better, preventing frequency
trajectories from curving too much and loosing stability. However,
a large value for this parameter will slow down the analysis significantly.
\item Minimal Segment Length (Frames): once the analysis is done, the spectral
data can be further \char`\"{}cleaned\char`\"{} up during post-processing.
Trajectories shorter than this value are suppressed if their average
SMR is below Minimal Segment SMR (see parameters 16 and 14). This
might help to avoid non-relevant sudden changes while keeping a high
frame rate, reducing also the number of intermittent sinusoids during
synthesis.
\item Minimal Gap Length (Frames): as parameter 11, this one is also used
to clean up the data during post-processing. In this case, gaps (zero
amplitude values, i.e. theoretical \char`\"{}silence\char`\"{}) longer
than Length frames are filled up with amplitude/frequency values obtained
by linear interpolation of the adjacent active frames. This parameter
prevents sudden interruptions of stable trajectories while keeping
a high frame rate.
\item SMR Threshold (dB SPL): also a post-processing parameter, the SMR
Threshold is used to eliminate partials with low averages.
\item Minimal Segment SMR (dB SPL): this parameter is used in combination
with parameter 11. Short segments with SMR average below this value
will be removed during post-processing.
\item Last Peak Contribution (0 to 1): as explained in Parameter 10, the
Peak Continuation Algorithm \char`\"{}looks-back\char`\"{} several
number of frames to do its job better. This parameter will help to
weight the contribution of the first precedent peak over the others.
A zero value means that all precedent peaks (to the size of Parameter
10) are equally taken in account.
\item SMR Contribution (0 to 1): In addition to the proximity in frequency
of the peaks, the ATS Peak Continuation Algorithm may use psycho-acoustic
information (the Signal-to-Mask-Ratio, or SMR) to improve the perceptual
results. This parameter indicates how much the SMR information is
used during tracking. For instance, a value of .5 makes the Peak Continuation
Algorithm to use a 50\% of SMR information and a 50\% of Frequency
Proximity information to decide which is the best candidate to continue
a sinusoidal track.
\end{enumerate}

\subsection{Transformation}


\subsubsection{Viewing Data}

\begin{enumerate}
\item Sinusoidal View: It can be seen that the frequency of each partial
is represented on the vertical (Y) axis, Time (in frames) runs along
the horizontal (X) axis, and amplitude is represented with a color
value. The two horizontal scrollbars control the time (frame) view.
The top one controls the from-view value, and the bottom one controls
the size of the view. There are three vertical scrollbars as well.
The two left-most ones control the frequency view (in a similar way
the horizontal scrollbars control the time view), and the right-most
scrollbar controls a contrast value for the amplitude display. Horizontal
and vertical scrollbars can be used to select and zoom in/out zones
of the spectral data. The contrast slider adjusts partials amplitude
display: a value of 50 shows the normal contrast between loud and
quiet partials, while a value of 100 overrides amplitude information
(i.e. all partials are displayed black). A value of 0 shows only very
loud partials. If the mouse pointer is on the image, the frame, time,
and frequency values of its position are printed on the status bar
on the bottom of the window.
\item Noise View: in order to view this, the analysis file must contain
Noise data. The energy value of each of the 25 Critical Bands (in
Bark scale) is shown as a color value along frames. If the mouse pointer
is on the image, the frame, time, frequency and energy values of its
position are printed on the status bar.
\item You can display the file header data by choosing view->file header.
\end{enumerate}

\subsubsection{Selecting Data}

To make any changes, the user must select some data. ATSH performs
both, a horizontal (frame) and a vertical (partial) selection. There
are four ways to select spectral data:

\begin{enumerate}
\item Using selection presets from the Edit menu. There are Select All,
Unselect All, Select Even, Select Odd, and Invert Selection routines.
\item Using the mouse. Block selection: When the left button is pressed,
the position of the mouse pointer at the first click represents the
first corner of a rectangular selection, and the position of the mouse
at the second click the diagonally-opposed corner of it. Single selection:
When the right button is pressed, the partial having the closer frequency
at the location of the pointer is selected if it previously was selected,
or vice-versa. If a block selection was previously done, and the pointer
of the mouse is in the selection rectangle, the other selected partials
remains selected. Otherwise, the selection is replaced by the new
selected partial for all the extent of the view. The data selected
will be displayed using red color.
\item Using the List View window (menu View). In the view menu all the data
can be seen under the form of a numerical list. The amplitude, frequency
and phase (if any) values of each frame are represented at each page
of the list. A vertical selection/deselection can be performed shift-clicking
/ ctrl-clicking on the list (note that you may also perform a non-contiguous
selection of partials as well). The horizontal selection may be done
using the NOW=TO and NOW=FROM buttons.
\item Using the smart selection menu item. This menu allows the user to
select partials over the lenght of the current view using both amplitude
evaluation and/or a fix step of partial order. As an example: setting
from = 1, to =10, jump by =2 and Amp. Threshold =-36 will select partials
1, 3, 5, 7 and 9 only if their amplitude (Peak or RMS) is above -36dB.
\end{enumerate}

\subsubsection{Editing Data}

At present, the only ways to change the selected data are to vary
the amplitude or the frequency. This is accomplished by applying an
envelope (linear or spline) to the amplitude or frequency values of
the partials over the selected time region. The envelopes can either
scale the data by multiplying the envelope values, or offset the data
by adding the envelope values.

If the resulting frequency values are greater than the maximal frequency
value present in the file, they will be truncated to this value. Also,
if the frequency values are changed, the phase information (if any)
is suppressed.

There are an unlimited number of {}``undos'' available to the user
as well as the possibility for editng multiple files by running several
instances of ATSH.


\subsection{Synthesis}

In order to synthesize data, an output soundfile name must be entered.
The soundfile format will be determined from the file extension. 

The user may choose to synthesize the sinusoidal part, the noise part(if
any), or a mix of both by manipulating the amplitudes.

Several features concerning synthesis may be set on the Synthesis/Parameters
menu. The user may scale the overall amplitude and frequency of the
original data using scalars. Note also that synthesis may use all
the data, or just a selection (if any). At present, ATSH's sinusoidal
synthesis engine is implemented as an array of linear interpolating
table-lookup oscillators. The noise part is synthesized by injecting
interpolated-noise modulation to each partial according to the energy
of the residual found in each Critical Band at the corresponding analysis
frame.

It is possible also to use a time function which allows the user to
stretch or to expand the file dynamically as well as read it forward
or backwards. The duration of the output file is represented on the
X (horizontal) axis while the temporal location of the data of the
analysis to be used in the synthesis is represented on the Y (vertical)
axis.
\end{document}
